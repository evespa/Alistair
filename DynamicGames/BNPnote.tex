\begin{frame}{Battaglini, Nunnari and Palfrey (2014)}
\begin{itemize}
\item Examine a dynamic public good problem 	\item Round payoff is given by 	$$ u_i(x^j_t,g_t)=x^j_t+\alpha \cdot \sqrt{g_t}$$ 	\begin{itemize} 	\item $g_t$ is the stock of the public good 	\item $x^j_t$ is private consumption 	\end{itemize} \pause 	\item Payoff for entire supergame is 	$$ \sum_{t=1}^{\infty} \delta^{t-1} \cdot u_i(x^j_t,g_t) $$ \end{itemize} \end{frame}
\begin{frame}{Battaglini, Nunnari and Palfrey (2014)} \begin{itemize} 	\item Initially public good has zero investment $g_0=0$ 	\item Each period, the agents receive income of $w$ which they allocate to 	\begin{itemize} 		\item $x^j_t$ private consumption 		\item $w-x^j_t$ contribution to public good 	\end{itemize}\pause 	\item The public good grows according to: 	$$ g_t=g_{t-1}+\sum^n_{j=1} (w-x^j_t)$$ \end{itemize} \end{frame}
\begin{frame}{Battaglini, Nunnari and Palfrey (2014)} 	\begin{itemize} 		\item Like a standard public goods problem, there is a static externality 		\begin{itemize} 			\item Internalize opportunity cost of contributing to public good 			\item Do not internalize public benefit 		\end{itemize}\pause 		\item But there is also a dynamic externality 		\begin{itemize} 			\item My contributions today will dissuade others from contributing in the future as public-good has diminishing returns 		\end{itemize} 	\end{itemize} \end{frame}
\begin{frame}{Battaglini, Nunnari and Palfrey (2014)} 	\begin{itemize} 		\item A social planner would have all agents fully invest in the public good until the marginal costs of further contributions equal the marginal gains 		\item So path has full investment until socially desireable level $y^\star_P$ is reached 		$$	y^\star_P=\left(\dfrac{\alpha n}{2(1-\delta)}\right)^2$$ 	\end{itemize} \end{frame}
\begin{frame}{Battaglini, Nunnari and Palfrey (2014)} 	\begin{itemize} 		\item Individuals will instead seek to maximize their own payoffs 		\item Statically, marginal gain to investment by the individual has return $\tfrac{1}{n}$th the size of planner's \pause 		\item But individuals will also have to account for how their behavior affects others future behavior\pause 		\item Plus they may be able to clawback their previous contributions 	\end{itemize} \end{frame}
\begin{frame}{Battaglini, Nunnari and Palfrey (2014)} 	\begin{itemize} 		\item Look at the case where agents investments in the public good are constrained so that 		$$ x\in\left[0,w+\tfrac{g}{n}\right] $$ 		\item Call this reversible investment case, as agents can remove their share of the public good $\tfrac{g}{n}$. 	\end{itemize} \end{frame}
\begin{frame}{Battaglini, Nunnari and Palfrey (2014)} 	\begin{itemize} 		\item Looking at symmetric, continuous Markov-perfect equilibria 		\begin{itemize} 			\item Want strategies to only depend on last-period stock $g$ 			\item Look for solutions where the individual's value function is concave 		\end{itemize}\pause 		\item Optimal path has stock of public good following the transition 			$$g_t=\min\left\{n\cdot w+g_{t-1} , \left(\dfrac{\alpha n}{2(n-\delta)}\right)^2  \right\}<y^\star_P $$ 	\end{itemize} \end{frame}
\begin{frame}{Battaglini, Nunnari and Palfrey (2014)} 	\begin{itemize} 		\item Contrast this with the case where investment in the public-good is irreversible, so 		$$x\in\left[0,w\right]$$ \pause 		\item There is now no fear that your own contributions will be expropriated by others 		\pause 		\item Steady-state investment level is now given by $$y^\star_I=\left(\dfrac{\alpha}{2(1-\delta)}\right)^2<y^\star_P$$ 	\end{itemize} \end{frame}
\begin{frame}{Battaglini, Nunnari and Palfrey (2014)} 	\begin{itemize} 		\item So Markov-perfect equilibria are not efficient, but other SPE might be\pause 		\item The MPE results have the steady-state order 		$$ y^\star_P>y^\star_I>y^\star_R>0$$ 		but the best-case SPE are different\pause 		\item Reversible investment allow for conditional punishments! 		\begin{itemize} 		\item For $\delta$ large enough can sustain the planner's solution through a Markov-trigger 	\end{itemize}\pause 	\item But with irreversible investment, deviations from the optimal path cannot be punished this way
	\end{itemize} \end{frame}
\begin{frame}{Battaglini, Nunnari and Palfrey (2014)} 	\begin{itemize} 		\item Caltech students 		\item Groups of $3$ or $5$ 		\item $\delta=\tfrac{3}{4}$ is probability of another round 		\item Receive $20$ (16) units per period in 3 (5) person treatment 		\item $\alpha=4$ 		\item Irreversible or Reversible environment 	\end{itemize} \end{frame}
\begin{frame}{Treatment Design} \begin{center} 	\includegraphics<1>[width=0.6\textwidth]{./img/BNPtbl1.pdf} \end{center} 	\begin{itemize} 	\item No effect from $n$ in  IIE 	\item Planner> IIE > RIE 	\end{itemize} \end{frame}
\begin{frame}{Results in round 10} \begin{center} 	\includegraphics<1>[width=0.6\textwidth]{./img/BNPtbl2.pdf} \end{center} 	\begin{itemize} 	\item Median level in round 10 (conditional on reaching it) 	\item Planner> IIE > RIE 	\end{itemize} \end{frame}
\begin{frame}{Median stock $g$ through time ($n=3$)} \begin{center} 	\includegraphics<1>[width=0.6\textwidth]{./img/BNPfig1a.pdf} 	\includegraphics<2>[width=0.6\textwidth]{./img/BNPfig1b.pdf} \end{center} 	\begin{itemize} 	\item Both treatments inefficient (unsurprising really) 	\item IIE is much greater, close to full contribution initially 	\item RIE drops off through time 	\end{itemize} \end{frame}
\begin{frame}{Median stock $g$ through time ($n=5$)} \begin{center} 	\includegraphics<1>[width=0.6\textwidth]{./img/BNPfig4a.pdf} 	\includegraphics<2>[width=0.6\textwidth]{./img/BNPfig4b.pdf} \end{center} 	\begin{itemize} 	\item Same patterns in $n=5$ treatment 	\end{itemize} \end{frame}
\begin{frame}{Median individual investments through time} \begin{center} 	\includegraphics<1>[width=0.6\textwidth]{./img/BNPfig2a.pdf} 	\includegraphics<2>[width=0.6\textwidth]{./img/BNPfig2b.pdf} \end{center} 	\begin{itemize} 	\item Investment drops off with time\pause 	\item Overinvestment relative to MPE prediction in both 	\end{itemize} \end{frame}
\begin{frame}{Distribution of Investment Types} \begin{center} 	\includegraphics<1>[width=0.6\textwidth]{./img/BNPtbl3-1.pdf} 	\includegraphics<2>[width=0.6\textwidth]{./img/BNPtbl3-2.pdf} \end{center} 	\begin{itemize} 	\item RIE initially positive, but then many begin taking some\pause 	\item IIE many more contribute full amount initially, drop to lower bound in later rounds 	\end{itemize} \end{frame}
\begin{frame}{Test of Markov} 	\begin{itemize} 	\item Add a static version of the problem 	\item Use the contination value  $v_R(g)$ for the reversible investment dynamic game 	\item Subjects asked to choose contribution $w-x$ where their payoff is 		$$ x_j+ \alpha \sqrt{\left(g_0+n\cdot w-\sum_{j=1}^n x_j\right)}+v_R\left(g_0+n\cdot w-\sum_{j=1}^n x_j \right) $$ 	\end{itemize} \end{frame}
\begin{frame}{Distribution of Investment Types} \begin{center} 	\includegraphics<1>[width=0.6\textwidth]{./img/BNPtbl4.pdf} \end{center} 	\begin{itemize} 	\item Each round given a randomly drawn value of stock $g_0$ 	\end{itemize} \end{frame}
\begin{frame}{Distribution of Investment Types} \begin{center} 	\includegraphics<1>[width=0.6\textwidth]{./img/BNPfig3a.pdf} 	\includegraphics<2>[width=0.6\textwidth]{./img/BNPfig3b.pdf} \end{center} 	\begin{itemize} 	\item For $n=3$ results pretty consistent\pause 	\item Similarly for $n=5$ 	\item IIE, many more contribute full amount initially, drop to lower bound in later rounds 	\end{itemize} \end{frame}
\begin{frame}{Classification of subjects} \begin{center} 	\includegraphics<1>[width=0.6\textwidth]{./img/BNPtbl5.pdf} \end{center} 	\begin{itemize} 	\item Conditional cooperation is those contributing a positive amount who respond positively to the contribution of others last round 	\end{itemize} \end{frame}
