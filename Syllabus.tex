\documentclass[11pt,english]{article}
\usepackage[unicode=true]{hyperref}
\usepackage{fullpage}
\usepackage[noamsmath]{kpfonts}
\usepackage{inconsolata}
\usepackage[T1]{fontenc}

\begin{document}
\title{Experimental Economics}
\author{Alistair J. Wilson \& Lise Vesterlund}
\date{Fall 2020}
\maketitle

\section*{Course Description}

Economics 2220 is a survey course of topics in experimental economics. The aim is to present some of the areas of research and to provide a foundation for original research in the field. Alistair Wilson will provide an introduction to the experimental methodology, then proceeding to review the literature on repeated/dynamic games, strategic communication, and political economy. Lise Vesterlund will be reviewing the literature on public and labor economics, as well as discussing the relative merits of laboratory versus field based experiments. An effort will be made to concentrate on series of experiments, in order to see how experiments build on one another and allow researchers with different theoretical dispositions to narrow the range of potential disagreement. The course is designed to familiarize the student with experimental methodology and the range of application of experimental methods in economic. Importantly the course will help the student work through an experimental project in experimental economics from idea, to design and coding the experiment, to collecting the data. While the course counts towards the two-course requirement for a major in experimental economics, it may also be taken as a standalone course.

\section*{Course Requirements}
Readings will be assigned before each class meeting and students are expected to have read the assigned papers before class and be able to summarize the readings when called upon. In addition, students will be asked to complete homeworks which familiarize them with applications of the methods examined in the class, primarily these will be data-driven exercises using the STATA package. 

The requirements for the course are active class participation, completing the assigned data-analysis homeworks, presentations in class, and ongoing contribution towards a final replication/methodology project, with the aim being to submit a jointly authored paper (among those actively engaged with the project) to a journal. All students are expected to submit either an assigned independently authored results-section for the joint project at the end of the semester, or to have conducted a full and detailed literature review on a topic to be discussed with the instructors.

\section*{Online Synchronicity:}
All lectures will be initially offered via zoom for synchronous consumption. If any students have extenuating circumstances that require them to be asynchronous, they should contact the instructors.

\section*{In-person classes:}
A room has been allocated to the class, such that if the university's risk posture allows we can move to in person classes. However, such a move will also only be made in consultation with the students and instructors. 

\section*{Additional Information:}
\subsection*{Course Website}
\href{http://www.pitt.edu/~alistair/}{Alistair Website}
and \href{http://www.pitt.edu/~vester/}{Lise Website}

\subsection*{Lecture Locations }

1540 Wesley W. Posvar Hall 

\subsection*{Contact Details}
\begin{itemize}
    \item \href{http://www.pitt.edu/~alistair/}{Alistair Wilson}, Dept. of
    Economics, University of Pittsburgh
    
        \begin{itemize}
        \item Office Location: 4915 Wesley W. Posvar Hall
        \item Telephone: 412-383-8152
        \item Email: alistair@pitt.edu
        \end{itemize}
    \item \href{http://pitt.edu/~vester}{Lise Vesterlund}, Dept. of Economics,
    University of Pittsburgh
    
        \begin{itemize}
        \item Office Location: 4926 Wesley W. Posvar Hall
        \item Telephone: 412-648-1794
        \item Email: vesterl@pitt.edu
        \end{itemize}
\end{itemize}

\subsection*{Office Hours}
Office hours are by appointment.

\subsection*{Academic Integrity}
Students in this course will be expected to comply with the University of Pittsburgh's \href{http://www.pitt.edu/~provost/ai1.html}{Policy on Academic Integrity}. Any student suspected of violating this obligation for any reason during the semester will be required to participate in the procedural process, initiated at the instructor level, as outlined in the University Guidelines on Academic Integrity. This may include, but is not limited to, the confiscation of the examination of any individual suspected of violating University Policy. Furthermore, no student may bring any unauthorized materials to an exam, including dictionaries and programmable calculators.

\subsection*{Disability Information}
If you have a disability that requires special testing accommodations or other classroom modifications, you need to notify both the instructor and the \href{http://www.drs.pitt.edu/policies.html}{Disability Resources and Services} no later than the 2nd week of the term. You may be asked to provide documentation of your disability to determine the appropriateness of accommodations. To notify Disability Resources and Services, call 648-7890 (Voice or TTD) to schedule an appointment. The Office is located in 140 William Pitt Union.

\subsection*{Statement on Classroom Recording}
To ensure the free and open discussion of ideas, students may not disseminate any recordings of the classroom lectures, the discussion, and/or project activities without the advance written permission of the instructors, where any such recordings (provided at each instructor's discretion) can be used solely for the student's private use.
\end{document}

